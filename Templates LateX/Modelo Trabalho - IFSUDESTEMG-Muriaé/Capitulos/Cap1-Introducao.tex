\chapter{Introdução} \label{Introducao}

Este documento de requisitos tem como objetivo definir e descrever as funcionalidades e características do sistema a ser desenvolvido, denominado Sistema de Pedidos estilo iFood. Ele servirá como guia para os desenvolvedores, ajudando-os a compreender os requisitos essenciais e as expectativas do cliente.

\section{Público-Alvo}

\dangersign[5ex]  {\color{red}
Na engenharia de software, o termo "público-alvo" na documentação de requisitos refere-se ao grupo específico de usuários, clientes ou stakeholders para os quais o software está sendo desenvolvido. Este grupo pode incluir diferentes tipos de usuários com diferentes necessidades, habilidades e expectativas em relação ao sistema. Identificar e entender o público-alvo é crucial para o sucesso do projeto de software, pois ajuda os desenvolvedores a direcionarem seus esforços para atender às necessidades e preferências desses usuários específicos. Isso pode ajudar na definição de requisitos funcionais e não funcionais que são relevantes para o público-alvo, bem como a criação de interfaces e experiências de usuário que sejam intuitivas e adequadas para esse grupo específico de usuários.}


O público-alvo deste documento inclui:

\begin{enumerate}
    \item \textbf{Equipe de Desenvolvimento:} Desenvolvedores, programadores e outros membros da equipe responsáveis pela implementação do sistema.
    
    \item \textbf{Gestores de Projeto:} Gerentes de projeto e líderes de equipe que supervisionam o desenvolvimento do sistema e garantem que os requisitos sejam atendidos.
    
    \item \textbf{Clientes e Usuários Finais:} Clientes que encomendaram o desenvolvimento do sistema e os usuários finais que irão interagir com ele após a conclusão.
    
    \item \textbf{Stakeholders:} Qualquer parte interessada no projeto, como investidores, patrocinadores e partes envolvidas no processo de desenvolvimento e implantação.
\end{enumerate}