\chapter{Descrição Geral do Produto} \label{cap:descricaogeral}

\section{Descrição do Cliente}

\dangersign[5ex]  {\color{red}
Requisitos de Usuário:
São requisitos de alto nível, escritos pelos próprios usuários.
Normalmente, são expressos em linguagem natural e não entram em detalhes técnicos.
Descrevem as necessidades e expectativas dos usuários em relação ao sistema.
Exemplos incluem funcionalidades essenciais, usabilidade, segurança e requisitos de documentação.
São a base para a definição dos requisitos de sistema.}

Atualmente lido com pedidos usando papel ou uma planilha no Excel. Estou procurando uma solução melhor e pensei em um sistema de pedidos online.

Quero algo fácil, onde clientes e meu restaurante possam se registrar rapidamente com seus dados pessoais.

Quero que os clientes naveguem pelo sistema de forma simples, vendo detalhes sobre menus, avaliações e fotos dos restaurantes.

Desejo que os clientes possam realizar seus pedidos online, ajustar itens, escolher entrega e adicionar observações. E os restaurantes precisam ser notificados rápido sobre novos pedidos.

Em relação a pagamentos, aceitar cartões e oferecer opções online.

Se houvesse um jeito de acompanhar pedidos online, seria perfeito, tanto para mim quanto para os restaurantes. Receber alertas sobre produtos quase acabando, isso ajudaria na gestão do estoque, e notificar clientes sobre promoções especiais seria uma vantagem.

Gostaria também de gerenciar estoque, cardápio, usuários, clientes, histórico de pedidos, contas a pagar e receber e entrada de produtos

 Aqui estão as principais funcionalidades desejadas para o sistema de pedidos online, com base nos requisitos do usuário:

\begin{itemize}
    \item \textbf{Registro Simplificado:} Permitir que clientes e restaurantes se registrem rapidamente, fornecendo apenas dados pessoais essenciais.
    
    \item \textbf{Navegação Intuitiva:} Facilitar a navegação dos clientes pelo sistema, permitindo que visualizem detalhes sobre menus, avaliações e fotos dos restaurantes de forma clara e organizada.
    
    \item \textbf{Realização de Pedidos Online:} Capacitar os clientes a fazerem pedidos de forma rápida e conveniente, com a possibilidade de ajustar itens, escolher opções de entrega e adicionar observações específicas.
    
    \item \textbf{Notificações Imediatas:} Garantir que os restaurantes sejam notificados instantaneamente sobre novos pedidos recebidos, permitindo uma resposta rápida e eficiente.
    
    \item \textbf{Opções de Pagamento Diversificadas:} Aceitar pagamentos via cartões de crédito e débito, além de oferecer opções de pagamento online para maior comodidade dos clientes.
    
    \item \textbf{Acompanhamento de Pedidos Online:} Implementar um sistema de acompanhamento de pedidos online para clientes e restaurantes, permitindo que ambos possam verificar o status atual dos pedidos em tempo real.
    
    \item \textbf{Alertas de Estoque:} Enviar alertas automáticos aos restaurantes sobre produtos com baixo estoque, auxiliando na gestão eficiente do estoque e evitando interrupções no atendimento.
    
    \item \textbf{Notificações de Promoções Especiais:} Enviar notificações aos clientes sobre promoções especiais, incentivando vendas adicionais e aumentando o engajamento.
    
    \item \textbf{Gestão Integrada:} Fornecer ferramentas abrangentes para gerenciar estoque, cardápio, usuários, clientes, histórico de pedidos, contas a pagar e receber, além de facilitar o processo de entrada de novos produtos no sistema.

    \item \textbf{Segurança e Confidencialidade:} A segurança dos dados de nossos clientes é uma prioridade. Portanto, precisamos garantir que o sistema seja seguro e protegido contra acessos não autorizados.
    
\end{itemize}

\section{Descrição Geral do Produto}

\dangersign[5ex]  {\color{red} A Descrição Geral do Produto é um componente importante da documentação de requisitos de software. Ela fornece uma visão ampla e abrangente do produto que está sendo desenvolvido. Geralmente, inclui informações sobre a finalidade do produto, suas principais características e funcionalidades, o contexto em que será usado, os principais stakeholders envolvidos, requisitos de desempenho, restrições de hardware ou software, entre outros detalhes relevantes.

Essa descrição serve como um ponto de referência para todos os envolvidos no desenvolvimento do produto, garantindo uma compreensão comum sobre o que está sendo criado e qual é o objetivo final. Além disso, ela ajuda a orientar o processo de desenvolvimento, facilitando a comunicação entre as equipes e garantindo que todas as partes interessadas tenham uma visão clara do produto desde o início do projeto.}


O Sistema de Delivery, denominado iFoodClone, é uma plataforma inovadora projetada para facilitar a experiência de pedidos online, inspirada no iFood. Este sistema possui como objetivo proporcionar uma eficiente forma para clientes consultar restaurantes e realizar pedidos personalizados, e acompanharem o status de suas entregas.

\subsection{Situação Atual da Empresa}

A empresa enfrenta desafios significativos em seu processo de delivery de alimentos devido à ausência de um sistema dedicado. Abaixo estão os principais pontos que caracterizam a situação atual:

\subsection*{\textit{Processo Manual}}

\begin{itemize}
    \item Registro manual de pedidos por telefone, levando a possíveis erros de comunicação.
    \item Dependência de anotações físicas para gerenciamento de pedidos.
    \item Processo demorado e suscetível a falhas humanas.
\end{itemize}

\subsection*{\textit{Limitações no Rastreamento de Pedidos}}

\begin{itemize}
    \item Falta de uma plataforma centralizada para rastreamento em tempo real.
    \item Dificuldade em informar os clientes sobre o status exato de seus pedidos.
    \item Possíveis atrasos na entrega devido à falta de visibilidade no processo.
\end{itemize}

\subsection*{\textit{Gestão de Estoque Manual}}

\begin{itemize}
    \item Controle de estoque realizado de forma manual.
    \item Dificuldade em prever demanda e gerenciar níveis de estoque eficientemente.
    \item Risco de falta de produtos essenciais e excesso de itens em estoque.
\end{itemize}

\subsection*{\textit{Limitação na Comunicação com Clientes}}

\begin{itemize}
    \item Comunicação limitada com os clientes, principalmente presencial ou por telefone.
    \item Falta de uma plataforma online para consulta de cardápio, promoções e interação direta.
    \item Oportunidades perdidas para atrair clientes por meio de estratégias digitais.
\end{itemize}

\subsection*{\textit{Análise e Relatórios Manuais}}

\begin{itemize}
    \item Ausência de ferramentas analíticas para análise de dados.
    \item Dificuldade em gerar relatórios de vendas, avaliar desempenho e identificar tendências.
    \item Tomada de decisões estratégicas prejudicada pela falta de informações detalhadas.
\end{itemize}

\subsection*{\textit{Implementação Limitada de Promoções}}

\begin{itemize}
    \item Desafios na implementação e gestão de promoções e descontos.
    \item Falta de flexibilidade para atrair e reter clientes por meio de estratégias promocionais.
    \item Potencial perda de competitividade no mercado devido a essa limitação.
\end{itemize}

\subsection*{\textit{Comunicação Desatualizada em Tempo Real}}

\begin{itemize}
    \item Falta de atualizações em tempo real para clientes e restaurantes parceiros.
    \item Possibilidade de informações desatualizadas, causando desconforto e insatisfação.
    \item Necessidade de melhorar a eficiência na comunicação para aprimorar a experiência do usuário.
\end{itemize}

\subsection{Escopo do Produto - Módulos e Funcionalidades}

\dangersign[5ex]  {\color{red}
O escopo refere-se ao conjunto de funcionalidades, características e requisitos que estão incluídos em um projeto ou atividade específica. Ele define os limites e a extensão do trabalho a ser realizado, delineando o que será entregue e o que não será. O escopo é crucial para garantir que todas as partes interessadas tenham uma compreensão clara do que será realizado durante o projeto e ajuda a evitar desvios e alterações desnecessárias ao longo do caminho.}

Este documento descreve o escopo do produto para o sistema de pedidos estilo iFood, abrangendo as funcionalidades principais do aplicativo de pedidos para iOS e Android, a API REST de comunicação e o sistema de gestão de pedidos, bem com, suas funcionalidades associadas. O sistema visa proporcionar uma experiência completa para clientes e restaurantes, facilitando a realização e gerenciamento de pedidos online.

\subsubsection*{Aplicativos de Pedidos para iOS e Android:}

Os aplicativos serão desenvolvidos para as plataformas iOS e Android, permitindo que clientes realizem pedidos de forma intuitiva e personalizada. Além disso, oferecerão funcionalidades como registro, autenticação, criação de perfis, navegação por restaurantes, visualização detalhada de menus, realização de pedidos online e notificações em tempo real.
\begin{center}
\begin{longtable}{|p{4cm}|p{11cm}|}
\hline
\textbf{Funcionalidades} & \textbf{Descrição} \\
\hline
\endfirsthead

\multicolumn{2}{c}%
{{\tablename\ \thetable{} -- Continuação da Página Anterior}} \\
\hline
\textbf{Funcionalidades} & \textbf{Descrição} \\
\hline
\endhead

Registro e autenticação de usuários (clientes e restaurantes) & Permite que usuários se cadastrem e acessem suas contas de forma segura. \\
\hline
Criação de perfis com detalhes adicionais & Oferece a personalização de perfis com informações como endereço e contato. \\
\hline
Navegação intuitiva por uma lista de restaurantes & Facilita a busca e seleção de restaurantes disponíveis. \\
\hline
Visualização detalhada de menus, avaliações e fotos & Fornecer informações detalhadas sobre os restaurantes para decisões informadas. \\
\hline
Realização de pedidos online com personalização & Permite que clientes personalizem seus pedidos e os realizem online. \\
\hline
Escolha de opções de entrega e adição de observações & Facilita a escolha de opções de entrega e a inclusão de observações especiais. \\
\hline
Notificações imediatas para restaurantes sobre novos pedidos & Garante que os restaurantes sejam alertados rapidamente sobre novos pedidos. \\
\hline
Suporte a diversas formas de pagamento & Aceitação de diferentes formas de pagamento, como cartões de crédito e débito. \\
\hline
Módulo web para acompanhamento de pedidos e histórico de transações & Acesso online para clientes acompanharem seus pedidos e visualizarem histórico de transações. \\
\hline
\caption{Funcionalidades do Aplicativo de Pedidos para iOS e Android.}
\label{tab:app_pedidos}
\end{longtable}
\end{center}




\subsubsection*{API REST de Comunicação}
A API REST será responsável pela comunicação segura entre os aplicativos de pedidos e o sistema de gestão de pedidos. Ela gerenciará registros de usuários, receberá e processará novos pedidos, atualizará em tempo real o status dos pedidos e integrará formas de pagamento, utilizando o formato JSON para a transferência eficiente de dados.

\begin{center}
    

\begin{longtable}{|p{4cm}|p{11cm}|}
\hline
\textbf{Funcionalidades} & \textbf{Descrição} \\
\hline
\endfirsthead

\hline
\textbf{Funcionalidades} & \textbf{Descrição} \\
\hline
\endhead

Comunicação segura entre aplicativos e o sistema de gestão & Estabelece uma comunicação segura para transmitir dados entre os aplicativos e o sistema de gestão. \\
\hline
Gerenciamento de registros de usuários & Responsável pelo cadastro, atualização e remoção de registros de usuários. \\
\hline
Recebimento e processamento de novos pedidos & Recebe e processa pedidos enviados pelos aplicativos de pedidos. \\
\hline
Atualização em tempo real do status dos pedidos & Mantém os aplicativos informados sobre o status atualizado dos pedidos. \\
\hline
Integração de formas de pagamento e confirmação de transações & Garante a integração bem-sucedida de diferentes formas de pagamento e confirmação de transações. \\
\hline
Transferência de dados no formato JSON & Utiliza o formato JSON para a transferência eficiente e estruturada de dados. \\
\hline
\caption{Funcionalidades da API REST de Comunicação.}
\label{tab:api_rest}
\end{longtable}
\end{center}

\subsubsection*{Sistema de Gestão de Pedidos}
O sistema de gestão proporcionará um painel de controle intuitivo para restaurantes gerenciarem suas operações. Com funcionalidades como gerenciamento simplificado de estoque e cardápio, geração de relatórios de vendas e estatísticas, controle de clientes, histórico de pedidos e usuários, além de notificações sobre produtos em baixo estoque e envio de promoções especiais.

\begin{center}
    

\begin{longtable}{|p{4cm}|p{10cm}|}
\hline
\textbf{Funcionalidades} & \textbf{Descrição} \\
\hline
\endfirsthead

\hline
\textbf{Funcionalidades} & \textbf{Descrição} \\
\hline
\endhead

Painel de controle intuitivo para restaurantes & Fornece um painel fácil de usar para restaurantes gerenciarem suas operações. \\
\hline
Gerenciamento simplificado de estoque e cardápio & Permite o controle eficiente de estoque e atualização do cardápio. \\
\hline
Relatórios de vendas e estatísticas & Gera relatórios detalhados de vendas e estatísticas para análise. \\
\hline
Controle de clientes, histórico de pedidos e usuários & Oferece ferramentas para gerenciar clientes, histórico de pedidos e usuários do sistema. \\
\hline
Notificações sobre produtos em baixo estoque para restaurantes & Alerta os restaurantes sobre produtos em baixo estoque para uma gestão proativa. \\
\hline
Envio de promoções especiais para clientes & Permite o envio de promoções especiais diretamente para os clientes. \\
\hline
\caption{Funcionalidades do Sistema de Gestão de Pedidos.}
\label{tab:sis_gestao}
\end{longtable}
\end{center}

\section{Premissas}

\dangersign[5ex]  {\color{red}
Uma premissa é uma afirmação ou proposição que serve como base para um argumento ou raciocínio. Em argumentos lógicos, as premissas são usadas para sustentar a conclusão.

Na documentação de requisitos de software, uma premissa é uma declaração fundamental que estabelece uma condição ou suposição sobre o ambiente, as necessidades dos usuários, as restrições técnicas ou qualquer outra informação relevante para o desenvolvimento do software. Essas premissas são usadas como base para definir os requisitos do sistema e ajudam a orientar o desenvolvimento do software para atender às expectativas e necessidades dos stakeholders.}


\subsubsection*{Disponibilidade de Acesso à Internet}
Para utilizar o Sistema de Emissão de Pedidos, é necessário acesso à internet para acessar a aplicação web.

\subsubsection*{Equipamentos Compatíveis}
Os usuários devem possuir dispositivos compatíveis, como computadores, tablets ou smartphones, para acessar o sistema.

\subsubsection*{Treinamento dos Usuários}
Os usuários serão treinados no uso do sistema antes de sua implantação para garantir uma transição suave e eficaz.

\subsubsection*{Backup dos Dados}
Será realizado um backup regular dos dados do sistema para evitar perdas de dados em caso de falhas ou incidentes.

\subsubsection*{Atualizações de Segurança}
O sistema receberá atualizações regulares de segurança para garantir a proteção contínua dos dados contra ameaças cibernéticas.

\subsubsection*{Suporte Técnico}
Um serviço de suporte técnico estará disponível para auxiliar os usuários em caso de problemas ou dúvidas relacionadas ao sistema.

\subsubsection*{Colaboração dos Usuários}
É esperado que os usuários colaborem ativamente no uso do sistema, fornecendo feedback e reportando quaisquer problemas encontrados durante o uso.