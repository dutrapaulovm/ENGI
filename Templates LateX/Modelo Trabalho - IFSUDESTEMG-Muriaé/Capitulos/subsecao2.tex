\section{Requisitos de Usuário}

\subsubsection*{Cadastro e Autenticação:}

\begin{enumerate}[label=1.\arabic*]
    \item \textbf{Registro Simples:} O sistema deve permitir que usuários, tanto clientes quanto restaurantes, se cadastrem fornecendo apenas informações essenciais, como nome, e-mail e senha.

    \item \textbf{Perfil Personalizado:} Após o registro, os usuários devem poder criar perfis personalizados, adicionando detalhes como endereço e número de contato.

    \item \textbf{Acesso Seguro:} O sistema garantirá que a entrada no sistema seja segura, protegendo as informações pessoais dos usuários.

\end{enumerate}

\subsubsection*{Consulta de Restaurantes}

\begin{enumerate}[label=2.\arabic*]
    \item \textbf{Exploração Intuitiva:} Os clientes devem conseguir explorar facilmente uma lista de restaurantes disponíveis no sistema.

    \item \textbf{Informações Detalhadas:} Detalhes completos, como menu, avaliações e fotos, estarão disponíveis para ajudar os clientes a tomarem decisões informadas.

\end{enumerate}

\subsubsection*{Realização de Pedidos}

\begin{enumerate}[label=3.\arabic*]
    \item \textbf{Pedido Descomplicado:} Os clientes terão a facilidade de fazer pedidos online, personalizando itens, escolhendo opções de entrega e adicionando observações especiais.

    \item \textbf{Confirmação Instantânea:} Restaurantes receberão notificações imediatas sobre novos pedidos.

\end{enumerate}

\subsubsection*{Opções de Pagamento}

\begin{enumerate}[label=5.\arabic*]
    \item \textbf{Facilidade de Pagamento:} O sistema oferecerá suporte a diversas formas de pagamento, incluindo cartões de crédito, débito e opções de pagamento online.

\end{enumerate}

\subsubsection*{Módulo Web de Acompanhamento}

\begin{enumerate}[label=6.\arabic*]
    \item \textbf{Acompanhamento Online:} Um módulo web permitirá que os clientes acompanhem o status de seus pedidos e visualizem o histórico de transações.

\end{enumerate}

\subsubsection*{Notificações de Estoque e Promoções}

\begin{enumerate}[label=7.\arabic*]
    \item \textbf{Alertas aos Restaurantes:} Restaurantes receberão alertas sobre produtos em baixo estoque.

    \item \textbf{Ofertas Exclusivas:} Clientes serão notificados sobre promoções especiais.

\end{enumerate}

\subsubsection*{Gestão Integrada}

\begin{enumerate}[label=8.\arabic*]
    \item \textbf{Controle Simplificado:} Restaurantes terão acesso a um painel de controle intuitivo para gerenciar estoque, cardápio, relatórios de vendas e usuários.

    \item \textbf{Funcionalidades Ampliadas:} O sistema permitirá que restaurantes controlem clientes, histórico de pedidos, contas a pagar, contas a receber, entrada de produtos e gestão de usuários.

\end{enumerate}




\section{Acrescentando um arquivo tex na estrutura}

Como acrescentar uma nova subseção, utilizando um arquivo externo:

\begin{enumerate}
    \item crie um arquivo .tex (ex.: meuarquivo.tex)
    \item Se for um arquivo de capítulo:
        \subitem No arquivo Monografia.text acrescente a seguinte linha na ordem que deseja aparecer no texto: $\backslash include\{Capitulos/meuarquivo\}$
    \item Se for parte do texto:
        \subitem Inserir no arquivo onde se deseja continuar o seguinte comando:
            \subsubitem $\backslash input\{Capitulos/meuarquivo\}$
    
    
\end{enumerate}


{\color{blue}

Obs.: Este texto foi escrito no arquivo exemplo subsecao2.tex. 
Note que o arquivo é inserido em continuidade na página!

}
